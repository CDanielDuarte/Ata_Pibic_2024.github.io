% Options for packages loaded elsewhere
\PassOptionsToPackage{unicode}{hyperref}
\PassOptionsToPackage{hyphens}{url}
\PassOptionsToPackage{dvipsnames,svgnames,x11names}{xcolor}
%
\documentclass[
  letterpaper,
  DIV=11,
  numbers=noendperiod]{scrartcl}

\usepackage{amsmath,amssymb}
\usepackage{iftex}
\ifPDFTeX
  \usepackage[T1]{fontenc}
  \usepackage[utf8]{inputenc}
  \usepackage{textcomp} % provide euro and other symbols
\else % if luatex or xetex
  \usepackage{unicode-math}
  \defaultfontfeatures{Scale=MatchLowercase}
  \defaultfontfeatures[\rmfamily]{Ligatures=TeX,Scale=1}
\fi
\usepackage{lmodern}
\ifPDFTeX\else  
    % xetex/luatex font selection
\fi
% Use upquote if available, for straight quotes in verbatim environments
\IfFileExists{upquote.sty}{\usepackage{upquote}}{}
\IfFileExists{microtype.sty}{% use microtype if available
  \usepackage[]{microtype}
  \UseMicrotypeSet[protrusion]{basicmath} % disable protrusion for tt fonts
}{}
\makeatletter
\@ifundefined{KOMAClassName}{% if non-KOMA class
  \IfFileExists{parskip.sty}{%
    \usepackage{parskip}
  }{% else
    \setlength{\parindent}{0pt}
    \setlength{\parskip}{6pt plus 2pt minus 1pt}}
}{% if KOMA class
  \KOMAoptions{parskip=half}}
\makeatother
\usepackage{xcolor}
\setlength{\emergencystretch}{3em} % prevent overfull lines
\setcounter{secnumdepth}{-\maxdimen} % remove section numbering
% Make \paragraph and \subparagraph free-standing
\makeatletter
\ifx\paragraph\undefined\else
  \let\oldparagraph\paragraph
  \renewcommand{\paragraph}{
    \@ifstar
      \xxxParagraphStar
      \xxxParagraphNoStar
  }
  \newcommand{\xxxParagraphStar}[1]{\oldparagraph*{#1}\mbox{}}
  \newcommand{\xxxParagraphNoStar}[1]{\oldparagraph{#1}\mbox{}}
\fi
\ifx\subparagraph\undefined\else
  \let\oldsubparagraph\subparagraph
  \renewcommand{\subparagraph}{
    \@ifstar
      \xxxSubParagraphStar
      \xxxSubParagraphNoStar
  }
  \newcommand{\xxxSubParagraphStar}[1]{\oldsubparagraph*{#1}\mbox{}}
  \newcommand{\xxxSubParagraphNoStar}[1]{\oldsubparagraph{#1}\mbox{}}
\fi
\makeatother


\providecommand{\tightlist}{%
  \setlength{\itemsep}{0pt}\setlength{\parskip}{0pt}}\usepackage{longtable,booktabs,array}
\usepackage{calc} % for calculating minipage widths
% Correct order of tables after \paragraph or \subparagraph
\usepackage{etoolbox}
\makeatletter
\patchcmd\longtable{\par}{\if@noskipsec\mbox{}\fi\par}{}{}
\makeatother
% Allow footnotes in longtable head/foot
\IfFileExists{footnotehyper.sty}{\usepackage{footnotehyper}}{\usepackage{footnote}}
\makesavenoteenv{longtable}
\usepackage{graphicx}
\makeatletter
\def\maxwidth{\ifdim\Gin@nat@width>\linewidth\linewidth\else\Gin@nat@width\fi}
\def\maxheight{\ifdim\Gin@nat@height>\textheight\textheight\else\Gin@nat@height\fi}
\makeatother
% Scale images if necessary, so that they will not overflow the page
% margins by default, and it is still possible to overwrite the defaults
% using explicit options in \includegraphics[width, height, ...]{}
\setkeys{Gin}{width=\maxwidth,height=\maxheight,keepaspectratio}
% Set default figure placement to htbp
\makeatletter
\def\fps@figure{htbp}
\makeatother

\KOMAoption{captions}{tableheading}
\makeatletter
\@ifpackageloaded{caption}{}{\usepackage{caption}}
\AtBeginDocument{%
\ifdefined\contentsname
  \renewcommand*\contentsname{Table of contents}
\else
  \newcommand\contentsname{Table of contents}
\fi
\ifdefined\listfigurename
  \renewcommand*\listfigurename{List of Figures}
\else
  \newcommand\listfigurename{List of Figures}
\fi
\ifdefined\listtablename
  \renewcommand*\listtablename{List of Tables}
\else
  \newcommand\listtablename{List of Tables}
\fi
\ifdefined\figurename
  \renewcommand*\figurename{Figure}
\else
  \newcommand\figurename{Figure}
\fi
\ifdefined\tablename
  \renewcommand*\tablename{Table}
\else
  \newcommand\tablename{Table}
\fi
}
\@ifpackageloaded{float}{}{\usepackage{float}}
\floatstyle{ruled}
\@ifundefined{c@chapter}{\newfloat{codelisting}{h}{lop}}{\newfloat{codelisting}{h}{lop}[chapter]}
\floatname{codelisting}{Listing}
\newcommand*\listoflistings{\listof{codelisting}{List of Listings}}
\makeatother
\makeatletter
\makeatother
\makeatletter
\@ifpackageloaded{caption}{}{\usepackage{caption}}
\@ifpackageloaded{subcaption}{}{\usepackage{subcaption}}
\makeatother
\ifLuaTeX
  \usepackage{selnolig}  % disable illegal ligatures
\fi
\usepackage{bookmark}

\IfFileExists{xurl.sty}{\usepackage{xurl}}{} % add URL line breaks if available
\urlstyle{same} % disable monospaced font for URLs
\hypersetup{
  pdftitle={Fundamentos do NGS},
  colorlinks=true,
  linkcolor={blue},
  filecolor={Maroon},
  citecolor={Blue},
  urlcolor={Blue},
  pdfcreator={LaTeX via pandoc}}

\title{Fundamentos do NGS}
\author{}
\date{2024-06-03}

\begin{document}
\maketitle

\href{https://www.youtube.com/watch?v=uKCCIAhLFrU}{texto baseado neste
video do eduardo castan}

\section{introdução ao NGS / sequênciamento de Segunda
geração}\label{introduuxe7uxe3o-ao-ngs-sequuxeanciamento-de-segunda-gerauxe7uxe3o}

O NGS são algumas tecnicas de sequênciamento desenvolvidas e englobadas
por duas pricipais empresas, \texttt{thermo-ficher\ e\ ilumina}. Apesar
de se divergerem na tecnologia e patêntes, ambas as tecnologias possuem
o mesmo fluxo de trabalho/princípio de funcionamento.

\subsection{Fluxo de trabalho de NGS}\label{fluxo-de-trabalho-de-ngs}

´´´´ Preparo de biblioteca -\textgreater{} Amplificação -\textgreater{}
Sequênciamento -\textgreater{} Análise ´´´´

\section{Geração de biblioteca}\label{gerauxe7uxe3o-de-biblioteca}

Essa é uma etapa comum aos sequênciamentos de ácidos núcleicos, etapa a
qual é responsavel por preparar a amostra para que ela seja compatível a
tecnologia.

\section{Amplificação}\label{amplificauxe7uxe3o}

As tecnologias necessitam amplificar os fragmentos de DNA para que o
equipamento possa alcançar a sensibilidade necessária para a o
sequênciamento

\section{Sequênciamento}\label{sequuxeanciamento}

Está é a etapa a qual o equipamento consegue ler / destinguir as bases
em sequência correta que estão no genoma

\section{Analise}\label{analise}

Está é a etapa de análise dos dados gerados pelo sequênciamento

\section{preparo de biblioteca}\label{preparo-de-biblioteca}

\subsubsection{Gerar fragmentos}\label{gerar-fragmentos}

Existem uma limitação na tecnica, ela não consegue sequenciar fragmetos
muito grandes \textgreater{} 600pb. Apenas de 200-250pb. Por isso é
necessário fragmentar esse genôme, com enzimas, sonicação ou
\texttt{pcr}. Está pcr é uma super multiplex, que consegue pegar todos o
genes

\subsubsection{Adicionar adaptadores nas extremidades dos
fragmentos}\label{adicionar-adaptadores-nas-extremidades-dos-fragmentos}

é adicionado um dna dupla fita nas extremidades dos fragmentos, e são
ligados por DNAliagese

\subsubsection{Adicionar marcadore de
amostra}\label{adicionar-marcadore-de-amostra}

\begin{itemize}
\tightlist
\item
  thermofish = barcode
\item
  ilumina = index
\end{itemize}

vem nos adaptadores, e são sequencias expecíficas que serve para separar
as amostras mesmo fazendo uma reação em um tubo so.

\section{Amplificação}\label{amplificauxe7uxe3o-1}

Apartir desse momento as tecnologias de diferen em relação a
amplificação da amostra

\subsection{Ilumina}\label{ilumina}

A tecnica de amplificação da ilumina é baseada em uma
\texttt{pcr\ em\ ponte} uma forma de pcr a qual em um chip com uma
região complementar aos adaptadores presentes no fragmento, assim
fixando verticalmente o fragmento ao chip. Neste local é realizado a
amplificação do alvo com a pcr e como há diversas dessas regiões
complemntares espalhadas ao longo do chip, é criado uma ponte durante a
amplificação, tornando possível a detecção de florecencia de
nucleotídeos adaptados para emitir uma florecencia expecífica ao passo
que a pcr avança, permitindo assim ao sequênciamento pois a tecnologia
da ilumina se baseaia na emissão de um sinal de florecência. com este
método é possivel gerar um sinal forte o suficiente para ser detectável

\begin{figure}[H]

{\centering \includegraphics{ngs_files/mediabag/sAAAAASUVORK5CYII=.png}

}

\caption{PCR em ponte}

\end{figure}%

O fragmento de dna fica com o formato de ponte pois na superfície há
primeres complementares ás sequências dos adaptadores, adicionados na
parte de construção de bibliotecas com o primer ligado ao fragmento,
durante os ciculos de pcrs, serão gerados novos fragmentos presos por
pontes, as quais serão quebradas as ligações entre os primeres para ser
aplificadas novamente, aumentando a superfície de florecência gerando
assim \texttt{clusteres} de fragmentos.

\subsection{Sequênciamento}\label{sequuxeanciamento-1}

Na tecnolgia da ilumina, devido o processo da pcr em ponte, é gerado
fragmentos no sentido
\texttt{5\textquotesingle{}\ -\textgreater{}\ 3\textquotesingle{}} e
\texttt{3\textquotesingle{}\ -\textgreater{}\ 5\textquotesingle{}}, os
quais são sequenciados, porém um sentido de fita de cada vez. Uma vez
que temos os \textbf{clusters} é adicionados os nucleotídeos especiais,
eles possuem um floróro com as cores das bases, semelhante ao
sequenciamento de primeria geração, os quais eles florecem dentro dos
clusteres e param a plimerização pela enzima Uma vez que a florecência é
detectada, o bloqueio e a florecencia da base complementar a fita
sequênciada é excluida da reação, possibilidando assim a adição de uma
nova base e continuando a adição de nucleotídeos e sequenciando todo o
fragmento.

\begin{figure}[H]

{\centering \includegraphics{ngs_files/mediabag/NGS-300x300.jpg}

}

\caption{Florecência de clusteres}

\end{figure}%

\subsection{Amplificação -\textgreater{}
Thermo-fisher}\label{amplificauxe7uxe3o---thermo-fisher}

A tecnologia de amplificação da thermo fisher é baseada em também uma
pcr porém uma \texttt{pcr\ em\ emulção} esta pcr separa os fragmentos
por missela. ou seja a ideia é separar apenas um fragmento e coloca-lo
dentro de uma missela, junto com todos os materiasi necessários para sua
amplificação. Em tubo com milhoes de misselas, dentro de cada uma o
fragmento será amplificado preso a uma \texttt{bead} uma bolinha a qual
têm centenas de primeres presos na sua superfície.

\begin{figure}[H]

{\centering \includegraphics{ngs_files/mediabag/Pyrosequencing.png.webp}

}

\caption{Amplificação da thermo ficher}

\end{figure}%

\subsection{Seqênciamento -\textgreater{} Thermo
Fisher}\label{sequxeanciamento---thermo-fisher}

Todas essas beads serão inseridas em um chip com 11 milhões de poços, os
quais so cabem uma bead por poço. O sinal detectado pela maquina é a
variação de Ph que é criado quando um nucleotídeo é incorporado pela dna
polimeraze, sempre que acontece essa incorporação é liberado uma
molécula de pirofostato e um atomo de hideogenio. o quê é detectável
pela maquina. Isso é possivel pois a maquina consegue tira e botar um
nucleotídeo por vez

\begin{figure}[H]

{\centering \includegraphics{ngs_files/mediabag/molecular-mechanism-}

}

\caption{Incorporação de nucçeptídeo}

\end{figure}%%
\begin{figure}[H]

{\centering \includegraphics{ngs_files/mediabag/Ion-Torrent-The-bead.png}

}

\caption{Sequênciamento Thermo ficher}

\end{figure}%



\end{document}
